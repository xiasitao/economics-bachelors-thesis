% !TeX root = scaffold-10.tex
\renewcommand{\imagepath}{../10-intro/img}

\chapter{Introduction}
Gutenberg, Franklin, Edison: Bringing about fundamental technological change can serve you a spot in the minds of future generations. Already at young age, children learn about those who shaped today's technological world, and some might already be excited about who will be the next: Elon Musk? And later maybe even themselves? Despite being a controversial public figure, Musk's career is arguably an inspiration for many adolescents. Yet, success stories like his are probably not benefitting everyone equally. In light of the ubiquity of role models in media and coverage of their successes and failures, they should not be overlooked as a potential factor of influence when addressing questions of social immobility and inequality. This thesis explores potential correlations between role models' media coverage and the \gls{ses} of those admiring them by the use of \gls{nlp} techniques. An overview of the methods will be given in section~\ref{ch:this_thesis} after an introduction on role models, socioeconomic status, and \gls{nlp}.

\section{Role Models}
During childhood and adolescence, the formation of identity is a crucial aspect of life. A substantial part of this formation process is constituted by projecting cultural and social requirements, ideals, and inspirations onto one's own future~\autocite{mcadams_psychology_2001}. In this phase, adults acting as role models play an important role as mediators of identity, attitude, and aspirations~\autocite{hurd_role_2011-1, morgenroth_how_2015}\todo{maybe replace \textcite{morgenroth_how_2015}}. The term role model can be defined broadly as a person, or an adult in particular, deserving to be imitated. More narrow definitions do, however, differentiate between mentors, role models, and heroines and heroes based on the proximity and intensity of interaction with the person~\autocite{pleiss_mentors_1995}.

The impact of role models in general on children and adolescents has been researched with regard to a diverse set of aspects. In child development, role models can contribute to identity formation~\autocite{vecci_behavioural_2019}, influence mental health~\autocite{bird_impact_2012}, and improve resilience~\autocite{werner_resilience_1995}. Also, celebrity role models can play a role as substitutes in case of lacking parental investment~\autocite{cheung_idol_2012}. In terms of education and career, role models and their identity can influence achievement and aspirations~\autocite{zirkel_is_2002, herrmann_effects_2016,christiansen_television_1979}. Depending on the quality of their exposed characteristics, role models have also been shown to be closely related to positive and negative behavior, such as prosociality~\autocite{kosse_formation_2020}, substance abuse~\autocite{yancey_role_2002, hurd_negative_2009}, delinquency~\autocite{walters_someone_2016}, and violence~\autocite{hurd_role_2011}.

A key determinant of the positive influence role models can exert on adolescents is how much the role models can be related to. On the one hand, this relatability is determined by how much the identities of the role model and of those admiring them match, in particular in terms of gender~\autocite{marx_female_2002, herrmann_effects_2016,lockwood_someone_2006}, ethnicity~\autocite{marx_obama_2009}, as well as domain and interest~\autocite{lockwood_superstars_1997}. On the other hand, the accessibility and proximity of role models can be related to the quality and time perspective of outcomes of admiring a role model~\autocite{strasser-burke_who_2020, bird_impact_2012}.

In light of this dependency on accessibility and relatability, it is important to distinguish between mentors and celebrity role models. It has been shown in interventions that providing children and adolescents with a personal mentor has positive effects on prosociality and education~\autocite{kosse_formation_2020, falk_mentoring_2020, dubois_natural_2005,rhodes_agents_2002,heckman_understanding_2013}. In contrast, celebrity role models differ from mentors in lacking this proximity and personal accessibility. Despite the fact that they are almost exclusively perceived via media and the internet, the impact they have on those admiring them does still depend on perceived accessibility and relatability~\autocite{strasser-burke_who_2020, lockwood_superstars_1997} as well as connotation and sentiment conveyed by media coverage~\autocite{lines_villains_2001, adamson_female_2019}. A second difference is that, in contrast to externally provided mentors, celebrity role models are mostly self-selected by children and adolescents.
\todo{maybe add transition}

\section{Socioeconomic Status}
The goal of this thesis is to explore how these role model characteristics correlate to the \gls{ses} of those who admire them. Generally, \gls{ses} is understood as a person's status with respect to financial situation, power, prestige and privileges~\autocite{mueller_measures_1981}, as well as general life satisfaction~\autocite{wagner_german_2007}. Social resources such as the social network as well as network members' \gls{ses} can also be counted towards one's \gls{ses}~\autocite{campbell_social_1986}. \gls{ses} has implications for a diverse set of life aspects, such as health~\autocite{oakes_measurement_2003, braveman_social_2011}, development of identity and preferences~\autocite{bradley_socioeconomic_2002, kraus_social_2012}, as well as curiosity, education, and career~\autocite{broer_review_2019,tucker-drob_socioeconomic_2012,walpole_socioeconomic_2003}.

In child development, a parallel between the role models and \gls{ses} can be identified when considering family life as a role model setting: The amount and the quality of parental investments represent a channel through which \gls{ses} determines the formation of intelligence and skills~\autocite{cunha_technology_2007,falk_socioeconomic_2021,cobb-clark_parenting_2019}. Also, via to its correlation with family integrity, \gls{ses} has an impact on child development~\autocite{conger_socioeconomic_2010}.




\todo{mention language development~\autocite{arriaga_scores_1998,hoff_socioeconomic_2005}}


\section{Natural Language Processing}
\gls{nlp} comprises a diverse set of computational techniques for enabling computers to deal with human language~\autocite{jurafsky_speech_2008}. Over the last decade, \gls{nlp} algorithms have evolved from frequency-based approaches towards ever more sophisticated deep learning language models trained on huge datasets, granting access to insights from unstructured language data and taking context ever more into account~\autocite{vajjala_practical_2020}. With their ability to classify text, extract information, translate, and infer meaning~\autocite{jurafsky_speech_2008}, \gls{nlp} algorithms are also gaining momentum as a tool in economic research, e.g. in market analyses~\autocite{hoberg_product_2010}, management research~\autocite{kang_natural_2020}, analysis of economic literature~\autocite{kim_keyword_2021,jelveh_detecting_2014,lambert_identifying_2021}, and research on economic policy~\autocite{elshehawy_sascat_2022}.


\section{This Thesis}\label{ch:this_thesis}
Considering the aforementioned effects of both role models and socioeconomic status on human and social capital, such as through child development, behavior and prosociality, as well as education, scrutinizing on their interplay is promising also from the perspective of education and social economics.

\dots

The rest of this thesis is organized as follows. Chapter~\ref{ch:data}...
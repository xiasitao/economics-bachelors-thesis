% !TeX root = scaffold-10.tex
\renewcommand{\imagepath}{../10-intro/img}

\chapter{Introduction}

\section{Role Models}
During childhood and adolescence, the formation of identity is a crucial aspect of life. A substantial part of this formation process is constituted by projecting cultural and social requirements, ideals, and inspirations onto one's own future~\autocite{mcadams_psychology_2001}. In this phase, adults acting as role models play an important role as mediators of identity, attitude, and aspirations~\autocite{hurd_role_2011-1, morgenroth_how_2015}. The term role model can be defined broadly as a person, or an adult in particular, deserving to be imitated. More narrow definitions do, however, differentiate between mentors, role models, and heroines and heroes based on the proximity and intensity of interaction with the person~\autocite{pleiss_mentors_1995}.

The impact of role models in general on children and adolescents has been researched in regard to a diverse set of aspects. Influences of role models on child development have been researched in terms of identity formation~\autocite{vecci_behavioural_2019}, mental health~\autocite{bird_impact_2012}, and resilience~\autocite{werner_resilience_1995}. Also, celebrity role models can play a role as substitutes in case of lacking parental investment~\autocite{cheung_idol_2012}. Also, in terms of education and career, role models and their identity can influence achievement and aspirations~\autocite{zirkel_is_2002, herrmann_effects_2016}\todo{check herrmann}


\section{Socioeconomic Status}

\section{Natural Language Processing}

\section{This Work}


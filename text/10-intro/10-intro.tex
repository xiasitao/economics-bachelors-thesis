% !TeX root = scaffold-10.tex
\renewcommand{\imagepath}{../10-intro/img}

\chapter{Introduction}\label{ch:intro}
Gutenberg, Franklin, Edison: Bringing about fundamental technological change can serve you a spot in the minds of future generations. Already at young age, children learn about those who shaped today's lives, and some might already be excited about who will be the next: Elon Musk? And later maybe even themselves? Despite being a controversial public figure, Musk's career is arguably an inspiration for many adolescents. Yet, success stories like his are probably not benefitting everyone equally. In light of the ubiquity of role models in media and coverage of their successes and failures, they should not be overlooked as a potential factor of influence when addressing questions of social immobility and inequality. This thesis explores potential correlations between role models' media coverage and the \gls{ses} of those admiring them by the use of \gls{nlp} techniques. An overview of the approach chosen in this thesis will be given in section~\ref{ch:this_thesis}, after an introduction to role models, socioeconomic status, and \gls{nlp}.

\section{Role Models}\label{ch:role_models}
During childhood and adolescence, the formation of identity is a crucial aspect of life. A substantial part of this formation process is constituted by projecting cultural and social requirements, ideals, and inspirations onto one's own future~\autocite{mcadams_psychology_2001}. In this phase, adults acting as role models play an important role as mediators of identity, attitude, and aspirations~\autocite{hurd_role_2011-1, morgenroth_how_2015}\todo{maybe replace \textcite{morgenroth_how_2015}}. The term role model can be defined broadly as a person, or an adult in particular, deserving to be imitated. More narrow definitions do, however, differentiate between mentors, role models, and heroines and heroes based on the proximity and intensity of interaction with the person~\autocite{pleiss_mentors_1995}.

The impact of role models in general on children and adolescents has been researched with regard to a diverse set of aspects. In child development, role models can contribute to identity formation~\autocite{vecci_behavioural_2019}, influence mental health~\autocite{bird_impact_2012}, and improve resilience~\autocite{werner_resilience_1995}. Also, celebrity role models can play a role as substitutes in case of lacking parental investment~\autocite{cheung_idol_2012}. In terms of education and career, role models and their identity can influence achievement and aspirations~\autocite{zirkel_is_2002, herrmann_effects_2016,christiansen_television_1979}. Depending on the quality of their exposed characteristics, role models have also been shown to be closely related to positive and negative behavior, such as prosociality~\autocite{kosse_formation_2020}, substance abuse~\autocite{yancey_role_2002, hurd_negative_2009}, delinquency~\autocite{walters_someone_2016}, and violence~\autocite{hurd_role_2011}.

A key determinant of the positive influence role models can exert on adolescents is how much the role models can be related to. On the one hand, this relatability is determined by how much the identities of the role model and of those admiring them match, in particular in terms of gender~\autocite{marx_female_2002, herrmann_effects_2016,lockwood_someone_2006}, ethnicity~\autocite{marx_obama_2009}, as well as domain and interest~\autocite{lockwood_superstars_1997}. On the other hand, the accessibility and proximity of role models can be related to the quality and time perspective of outcomes of admiring a role model~\autocite{strasser-burke_who_2020, bird_impact_2012}.

In light of this dependency on accessibility and relatability, it is important to distinguish between mentors and celebrity role models. It has been shown in interventions that providing children and adolescents with a personal mentor has positive effects on prosociality and education~\autocite{kosse_formation_2020, falk_mentoring_2020, dubois_natural_2005,rhodes_agents_2002,heckman_understanding_2013}. In contrast, celebrity role models differ from mentors in lacking this proximity and personal accessibility. Despite the fact that they are almost exclusively perceived via media and the internet, the impact they have on those admiring them does still depend on perceived accessibility and relatability~\autocite{strasser-burke_who_2020, lockwood_superstars_1997} as well as connotation and sentiment conveyed by media coverage~\autocite{lines_villains_2001, adamson_female_2019}. A second difference is that, in contrast to externally provided mentors, it is in the nature of things that celebrity role models are mostly self-selected by children and adolescents.


\section{Socioeconomic Status}\label{ch:ses}
The goal of this thesis is to explore how these perceived role model characteristics correlate to the \gls{ses} of those who admire them. Generally, \gls{ses} is understood as a person's status with respect to financial situation, power, prestige and privileges~\autocite{mueller_measures_1981}, as well as general life satisfaction~\autocite{wagner_german_2007}. Social resources such as the social network as well as network members' \gls{ses} can also be counted towards one's \gls{ses}~\autocite{campbell_social_1986}. \gls{ses} has implications for a diverse set of life aspects, such as health~\autocite{oakes_measurement_2003, braveman_social_2011}, development of identity and preferences~\autocite{bradley_socioeconomic_2002, kraus_social_2012}, as well as curiosity, education, and career~\autocite{broer_review_2019,tucker-drob_socioeconomic_2012,walpole_socioeconomic_2003}. 

These implications show similarities to the effects of role models discussed before. In particular in child development, a parallel between the role models and \gls{ses} can be identified when considering family life as a role model setting: The amount and the quality of parental investments represent a channel through which \gls{ses} determines the formation of intelligence and skills~\autocite{cunha_technology_2007,falk_socioeconomic_2021,cobb-clark_parenting_2019}. Also, via to its correlation with family integrity, \gls{ses} has an impact on child development~\autocite{conger_socioeconomic_2010}.\todo{maybe add education}

Two areas of influence of \gls{ses} that are especially interesting for this thesis lie in language acquisition and newspaper consumption. In the early stages of life, low \gls{ses} has a pronounced negative impact on the development of language~\autocite{arriaga_scores_1998,hoff_socioeconomic_2005}. Later in life, \gls{ses} determines to some extent media consumption and in particular newspaper consumption through cultural capital and education~\autocite{york_youth_2015,lee_motivational_2014}. In particular, \gls{ses} has some correlation with the journalistic style that is preferably consumed~\autocite{ohlsson_matter_2017,bergstrom_towards_2019}. Furthermore, consumption of celebrity news is even discussed as a factor contributing to differences in \gls{ses} being upheld~\autocite{dubied_studying_2014}.


\section{Natural Language Processing}\label{ch:nlp}
\gls{nlp} comprises a diverse set of computational techniques for enabling computers to deal with human language~\autocite{jurafsky_speech_2008}. Over the last decade, \gls{nlp} algorithms have evolved from frequency-based approaches towards ever more sophisticated deep learning language models trained on huge datasets, granting access to insights from unstructured language data and taking context ever more into account~\autocite{vajjala_practical_2020}. With their ability to classify text, extract information, translate, and infer meaning~\autocite{jurafsky_speech_2008}, \gls{nlp} algorithms are also gaining momentum as a tool in economic research, e.g. in market analyses~\autocite{hoberg_product_2010}, management research~\autocite{kang_natural_2020}, analysis of economic literature~\autocite{kim_keyword_2021,jelveh_detecting_2014,lambert_identifying_2021}, and research on economic policy~\autocite{elshehawy_sascat_2022}.


\section{This Thesis}\label{ch:this_thesis}
As elaborated above, there are many channels through which role models and \gls{ses} both influence aspects of cultural and human capital, such as child development, behavior and prosociality, and education. Scrutinizing on their interplay is therefore promising from the perspective of educational and social economics, considering potential implications for inequality and social immobility.

The objective of this thesis is a data-driven exploration of how qualitative aspects of celebrity news reports correlate with the \gls{ses} of adolescents admiring them as role models. Taking the various channels of influence of both role models and \gls{ses} into account, the investigation will focus on news report topics and connotations, conveyed sentiment, perceivable relatability and proximity, prosociality and crime, as well as journalistic aspects of the reports. The goal is to identify any differences in news reports reports about role models of low-\gls{ses} and high-\gls{ses} adolescents. The key assumption justifying the use of news articles is that the adolescents are almost exclusively perceiving their role models through media, implying that any traits of the role models must be reflected in reports about them in order to be impactful.

This thesis is a follow-up of \textcite{fenske_using_2022} who quantified celebrity news report sentiment using \gls{nlp} methods and explored correlations with \gls{ses}. Even though a small positive correlation between news article sentiment and \gls{ses} was found, the results were not significant. This thesis continues on this track and explores other aspects besides sentiment.

The data used in the exploration is very similar to the data used by \textcite{fenske_using_2022} and consists of two datasets: survey data among German adolescents indicating their \gls{ses} and their self-selected celebrity role models, and online newspaper articles about these role models. The newspaper articles are analyzed using unsupervised and supervised \gls{nlp} techniques yielding qualitative information about topics, sentiments, and connotations.

The rest of this thesis is organized as follows. Chapter~\ref{ch:data} gives a thorough overview of the two data sources used for the analysis and discusses properties and shortcomings of the data. Chapter~\ref{ch:algorithms} then briefly introduces the \gls{nlp} algorithms used for the analyses. In chapter~\ref{ch:unsupervised}, the unsupervised \gls{nlp} approach for exploring article topics is explained in detail and differences between the \gls{ses} groups in terms of celebrity news topics are discussed. In chapter~\ref{ch:supervised}, the analyses will be directed more closely to connotations, sentiments, and journalistic style of the celebrity news reports by the use of supervised \gls{nlp} techniques, and results will be discussed. The last chapter concludes with a discussion of the findings and suggestions for enhancing data aggregation and \gls{nlp} techniques.
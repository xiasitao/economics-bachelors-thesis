% !TeX root = scaffold-10.tex
\renewcommand{\imagepath}{../10-intro/img}

\chapter{Introduction}

\section{Role Models}
During childhood and adolescence, the formation of identity is a crucial aspect of life. A substantial part of this formation process is constituted by projecting cultural and social requirements, ideals, and inspirations onto one's own future~\autocite{mcadams_psychology_2001}. In this phase, adults acting as role models play an important role as mediators of identity, attitude, and aspirations~\autocite{hurd_role_2011-1, morgenroth_how_2015}\todo{maybe replace \textcite{morgenroth_how_2015}}. The term role model can be defined broadly as a person, or an adult in particular, deserving to be imitated. More narrow definitions do, however, differentiate between mentors, role models, and heroines and heroes based on the proximity and intensity of interaction with the person~\autocite{pleiss_mentors_1995}.

The impact of role models in general on children and adolescents has been researched with regard to a diverse set of aspects. In child development, role models can contribute to identity formation~\autocite{vecci_behavioural_2019}, influence mental health~\autocite{bird_impact_2012}, and improve resilience~\autocite{werner_resilience_1995}. Also, celebrity role models can play a role as substitutes in case of lacking parental investment~\autocite{cheung_idol_2012}. In terms of education and career, role models and their identity can influence achievement and aspirations~\autocite{zirkel_is_2002, herrmann_effects_2016,christiansen_television_1979}. Depending on the quality of their exposed characteristics, role models have also been shown to be closely related to positive and negative behavior, such as prosociality~\autocite{kosse_formation_2020}, substance abuse~\autocite{yancey_role_2002, hurd_negative_2009}, delinquency~\autocite{walters_someone_2016}, and violence~\autocite{hurd_role_2011}.

A key determinant of the positive influence role models can exert on adolescents is how much the role models can be related to. On the one hand, this relatability is determined by how much the role model's and the admirer's identities match, in particular in terms of gender~\autocite{marx_female_2002, herrmann_effects_2016,lockwood_someone_2006}, ethnicity~\autocite{marx_obama_2009}, as well as domain and interest~\autocite{lockwood_superstars_1997}. On the other hand, the accessibility and proximity of role models can relate to the outcomes and time perspective of having a role model~\autocite{strasser-burke_who_2020, bird_impact_2012}.

In light of this dependency on accessibility and relatability, it is important to distinguish between mentors and celebrity role models. It has been shown in interventions that providing children and adolescents with a mentor has positive effects on prosociality and education~\autocite{kosse_formation_2020, falk_mentoring_2020, dubois_natural_2005,rhodes_agents_2002,heckman_understanding_2013}. In contrast to the close relationship in mentoring, it is in the nature of things that celebrity role models are distant from admirers and personally inaccessible. They are almost exclusively perceived via media and the internet.



\section{Socioeconomic Status}

\section{Natural Language Processing}

\section{This Work}


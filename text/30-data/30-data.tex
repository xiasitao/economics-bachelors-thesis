% !TeX root = scaffold-30.tex
\renewcommand{\imagepath}{../30-mot/img}

\chapter{Data Overview and Preprocessing}
For the analyses in this work, survey data of adolescents' celebrity role models and their socioeconomic status (SES) was used alongside online newspaper articles about these role models. In this chapter, each data set and its fundamental descriptive statistics are presented in details along with the respective preprocessing steps applied before the analyses.

\section{Adolescents' Socioeconomic Status and their Role Models}
The SES dimension of the data used in this work stems from a long-term study on the link between prosociality and SES started in 2011~\autocite{kosse_formation_2020}. This study had originally researched the influence of a year of student mentoring on the prosociality of second-graders from low-SES backgrounds living in German Northrhine-Westphalia. At the end of the original interview period, the cohort consisted of \SI{412}{children} from low- and \SI{97}{children} from high-SES backgrounds. A participant was labeled low-SES if their household had less than the \SI{30}{\percent} quantile of the German income distribution at the time, or none of their parents had a degree permitting university studies, or they were raised by a single parent~\autocite{kosse_formation_2020}.

\paragraph{Role Models}
In a follow-up round of interviews in 2018, the now 13- and 14-year-old adolescents were interviewed about their role models. At this stage, \SI{245}{children} of low- and \SI{96}{children} of high SES participated in the interview.

Every participant could name up to five role models, most of them, however, indicated just one. Not all answers were valid: only celebrities and groups of celebrities (e.g. bands) were counted, entire professions (``sportsmen/women'', ``YouTuber'') were excluded. In total, \SI{236}{} participants gave \SI{332}{} valid answers, resulting in a set of \SI{216}{} distinct role models (\SI{60}{} female, \SI{156}{} male).

\paragraph{Role Models and SES}
The role models were associated with SES status in two different ways:
\begin{itemize}
    \item In the \textit{general} approach, each role model is associated with low SES and high-SES if it was mentioned at least once by a participant in the respective SES group, resulting in \SI{172}{} low-, \SI{72}{} high-SES role models, and \SI{28}{} role models associated with both SES levels. Formally, this set $\mathcal{R}_\text{general}$ splits into $\mathcal{R}_\text{distinct, low}$ and $\mathcal{R}_\text{distinct, high}$ with
    \begin{align*}
        \left|\mathcal{R}_\text{distinct, low}\right| &= \SI{172}{},\\
        \left|\mathcal{R}_\text{distinct, low}\right| &= \SI{72}{},\\
        \left|\mathcal{R}_\text{distinct, low} \cap \mathcal{R}_\text{distinct, high}\right| &= \SI{28}{}.
    \end{align*}

    \item In the \textit{distinct-SES} approach, only role models were considered who were mentioned only by either low- or high-SES survey participants exclusively. Role models who were mentioned by both low- and high-SES participants are excluded from this set. This set contains \SI{144}{} role models associated with low SES and \SI{44}{} role models associated with high SES. Formally, this set $\mathcal{R}_\text{distinct}$ of distinct-SES role models splits into $\mathcal{R}_\text{distinct, low}$ and $\mathcal{R}_\text{distinct, high}$ with 
    \begin{align*}
        \left|\mathcal{R}_\text{distinct, low}\right| &= \SI{144}{},\\
        \left|\mathcal{R}_\text{distinct, low}\right| &= \SI{44}{}\\
        \left|\mathcal{R}_\text{distinct, low} \cap \mathcal{R}_\text{distinct, high}\right| &= \SI{0}{}.
    \end{align*}    
\end{itemize}

Note that, on average, each role model was mentioned only \SI{1.54}{times}. \SI{158}{} of them were mentioned only once, only four of them were mentioned four times or more often which means that most role models' SES association is hence not based on a lot of statistical variation. This association is hence not so well-grounded and must be dealt with carefully. The idea of filtering out role models who were just mentioned once, however, was decided against because then the number of high-SES role models would have become too small to say anything meaningful about this group.


\section{Newspaper Articles}
% !TeX root = scaffold-80.tex
\renewcommand{\imagepath}{../80-outro/img}

\chapter{Discussion and Outlook}
In this thesis, role models and \gls{ses} were introduced as phenomena correlating with a variety of factors of social and human capital in a broad sense, such as child development, education, and prosociality. Survey data capturing adolescents' \gls{ses} and self-selected role models as well as a dataset of online newspaper articles about these role models were presented as the basis for two \gls{nlp} analysis approaches for identifying differences in newspaper reports about the role models of low- and high-\gls{ses} adolescents.

\paragraph{Findings}
Significant differences in the distribution of topics in the reports about the low- and the high-\gls{ses} role models could be identified: sports- and music-related topics were more frequent in the low-\gls{ses} group while social life- and movie-related topics were more abundant in the high-\gls{ses} group of newspaper articles. A significant difference between the \gls{ses} groups was found for connotations of prosociality, namely that in the low-\gls{ses} antisocial-connotated articles are more prevailing than in the high-\gls{ses} group, and opposite for prosocial-connotated articles. Particularly interesting is the finding that drug-related articles are significantly more prevalent among the low-\gls{ses} than the high-\gls{ses} role models. Furthermore, tendencies towards negative sentiment in the low-\gls{ses} and towards neutral and positive sentiment in the high-\gls{ses} group were found but not highly significant. A similar non-significant tendency could be identified between low and high \gls{ses} and the connotations of failure and success respectively.

\paragraph{External Validity and Suggested Improvements}
The external validity of the findings must be judged in light of the data quality and the applied methods.

In the survey data, a large number of role models is named, the scope of their backgrounds is, however, concentrated on just a few professions that dominate the newspaper article topics, making it harder to access nuances of the texts. As far as the association between role models and \gls{ses} is concerned, a major threat is the small number of just \si[]{1.54} times each role model had been mentioned by a survey participant on average. This low ratio implies that it cannot be reliably stated whether and to what extent a role model pertains to the low- or the high-\gls{ses} groups. In this dataset, the effect of some role models being associated to the wrong group due to statistical noise might average out thanks to their large number and thus alleviate this problem. Increasing the metions-to-role model ratio would, however, lead to a  more robust \gls{ses}-association of the role models and allow a more well-grounded interpretation of \gls{ses} differences. This could be achieved though a restricted number of role models and having participants rank a given set of role models instead of naming any, or by asking a lot more adolescents for their role models and filtering out role models that were mentioned by just a few participants.

Even though adolescents get impressions of their celebrity role models almost exclusively through media and thus online celebrity news probably cover a good portion of how role models impact them, online newspaper articles might be missing important pieces of information that adolescents consume via their role models' social media accounts. Therefore, it could be advisable to enrich the textual corpus used for the analyses with social media data. Nevertheless, social media might suffer from being less objective than newspaper articles since it is published by the celebrities themselves who might intentionally suppress certain information, calling for increased caution when adding social media posts to the text corpus.

In terms of data processing, a more refined data aggregation strategy could be advisable in order to distill more concise information about the aforementioned significant differences between the \gls{ses} groups. One possibility would be aggregation on the role model level. The higher amount of drug-related reports in the low-\gls{ses} group, e.g., could be further scrutinized by aggregating the articles per role model and identifying individual role models reported about in a drug context. In a multi-stage approach, further insights could then be generated by comparing the reports about the drug and the non-drug role models. Also, correlations between the labels of different categories, as was mentioned for topic and article type, could bring further insights.

With respect to model choice and applications, fine-tuning models with labelled data, as it is possible for BERT-like models, could help reap the benefits of both general pre-training and problem-specific conditioning~\autocite{devlin_bert_2019}. To this end, more articles would need to be human-annotated, and the models could be enhanced with these annotations in a semi-supervised approach, as suggested by \textcite{fenske_using_2022}.


% Outlook
\paragraph{Outlook}
At the time this thesis was written, most of the survey participants were probably just finishing secondary education. But within a decade, almost all of them will find themselves pursuing their professional careers. At this point, taking the data out of the drawer again and having a look at how the role models accompanying their coming of age correlate with their professional future might give access to more profound insights on the interplay of role models and the formation of human capital. Knowing these influences can inspire society and institutions to the design of role model-related interventions that go beyond individual mentoring, e.g. by an evidence-based targeted integration into education curricula. More generally, researching the influence of adolescents' social surroundings in their entirety will probably become ever more feasible with the advent of ever more capable machine learning algorithms and ever more abundant stocks of data. From a long-term perspective, these galloping advances make the vision of anticipating and channeling the formation of human capital in future generations and thereby helping increase social mobility, seem less like an audacious vision rather than an achievable goal.
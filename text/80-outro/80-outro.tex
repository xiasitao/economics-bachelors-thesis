% !TeX root = scaffold-80.tex
\renewcommand{\imagepath}{../80-outro/img}

\chapter{Discussion and Outlook}
In this thesis, role models and \gls{ses} were introduced as phenomena correlating with a variety of factors of human capital, such as child development, education, and prosociality. Survey data capturing adolescents' \gls{ses} and self-selected role models as well as a dataset of online newspaper articles about these role models were presented as the basis for two \gls{nlp} analysis approaches for identifying differences in newspaper reports about the role models of low- and high-\gls{ses} adolescents.

\paragraph{Findings}
Significant differences in the distribution of topics in the reports about the low- and the high-\gls{ses} role models could be identified: sports- and music-related topics were more frequent in the low-\gls{ses} group while social life- and movie-related topics were more abundant in the high-\gls{ses} group of newspaper articles. A significant difference between the \gls{ses} groups was found for connotations of prosociality, namely that in the low-\gls{ses} antisocial-connotated articles are more prevailing than in the high-\gls{ses} group, and opposite for prosocial-connotated articles. Particularly interesting is the finding that drug-related articles are significantly more prevalent among the low-\gls{ses} than the high-\gls{ses} role models. Furthermore, tendencies towards negative sentiment in the low-\gls{ses} and towards neutral and positive sentiment in the high-\gls{ses} group were found but not highly significant. A similar non-significant tendency could be identified between low and high \gls{ses} and the connotations of failure and success respectively.

% Validity (see also outlook.md)
% - A lot of variance of individuals
% - No variation of ses per role model
% - Little variation of role model professions and backgrounds
% - Methods: aggregation rather straightforward and not so elaborate
The external validity of the findings must be judged in light of the data quality and the methods applied. 

\paragraph{Suggested Improvements}
% Improvements
% 
% - aggregation strategy (pronouncing more the role model level)
In order to distill more concise information about the aforementioned significant differences between the \gls{ses} groups, a more refined data aggregation strategy than the straightforward comparison of label distributions is advisable, such as aggregation on the role model level. The higher amount of drug-related reports in the low-\gls{ses} group, e.g., could be further scrutinized by aggregating the articles per role model and identifying role models reported about in a drug context. In a multi-stage approach, further insights could then be generated by comparing the reports about the drug and the non-drug role models. Also, correlations between the labels of different categories, as was mentioned for topic and article type, could bring further insights. As BERT-like models can be fine-tuned with labelled data in order to reap the benefits of both general pre-training and problem-specific conditioning~\autocite{devlin_bert_2019}, it could also be promising to human-annotate more articles and enhance the language models used with these annotations, similar to the approach of \textcite{fenske_using_2022}.


% Outlook
\paragraph{Outlook}
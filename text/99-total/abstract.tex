Role models have been shown to influence the life and development of children and adolescents admiring them in a wide range of aspects, such as identity formation, education, and prosociality. Similarly, \gls{ses} has an impact on these facets of life which suggests that role models could, in turn, be a factor of social mobility and inequality, making the interaction of role models and \gls{ses} a promising target of research. In this thesis, survey results capturing adolescents' \gls{ses} and self-selected celebrity role models are combined with natural language processing analyses of online newspaper articles about these role models. It aims at comparing how reports about the role models of low- and high-\gls{ses} individuals differ in terms of topic and conveyed connotations. Significant differences in the distribution of article topics and of connotations of sentiment, success, and prosocial behavior were found between the \gls{ses} groups.

% \vspace{2cm}
% Vorbildern wurde in vielerlei Hinsicht Einfluss auf die Entwicklung und das Leben von Kindern und Jugendlichen nachgewiesen, besonders auf die Entstehung von Identität, auf ihre Bildung und auf ihr prosoziales Verhalten. Auch der sozioökonomische Status (SES) ist ein bedeutender Einflussfaktor auf ebendiese Aspekte.
% Dies legt einen möglichen Zusammenhang zwischen Vorbildern und SES und somit einen Einfluss auf soziale Ungleichheit und soziale Mobilität nahe, was eine Untersuchung des Zusammenwirkens von Vorbildern und SES nahelegt.
% In dieser Arbeit werden Ergebnisse einer Befragung von Jugendlichen zu ihrem SES und ihren selbstgewählten prominenten Vorbildern mit computerlinguistischen Analysen von Online-Zeitungsberichten über diese Vorbilder kombiniert. Dabei wird verglichen, wie sich die Berichte über von Jugendlichen mit niedrigem und hohem SES benannten Vorbilder im Hinblick auf Themen und Konnotationen unterscheiden. Es wurden signifikante Unterschiede in den Verteilungen der Themen der Zeitungsartikel sowie der darin vermittelten Konnotationen zu Gefühlslagen, Erfolg, prosozialem Verhalten zwischen den niedrigem und hohem SES gefunden.

\glsresetall
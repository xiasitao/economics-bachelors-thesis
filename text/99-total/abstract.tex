Role models have been shown to affect child development with respect to identity formation, behavior and prosociality, as well as education and career. 
This influence suggests that the perception of role models may have repercussions on social mobility and inequality, given that \gls{ses} is related to similar facets of adolescent development.
From a perspective of social economics, interaction effects of \gls{ses} and role model perception are therefore a promising topic of research.
In this thesis, survey results capturing adolescents' \gls{ses} and self-selected celebrity role models are combined with natural language processing analyses of online newspaper articles about these role models.
The objective is to compare how reports about the role models of low- and high-\gls{ses} individuals differ in terms of topics and conveyed connotations.
Significant differences in the distribution of article topics are identified between the \gls{ses} groups. Positive connotations of sentiment, success, and behavior are found to be slightly more pronounced in high-\gls{ses} role model reports.

% \vspace{2cm}
% Vorbildern wurde in vielerlei Hinsicht Einfluss auf die Entwicklung und das Leben von Kindern und Jugendlichen nachgewiesen, besonders auf die Entstehung von Identität, auf ihre Bildung und auf ihr prosoziales Verhalten. Auch der sozioökonomische Status (SES) ist ein bedeutender Einflussfaktor auf ebendiese Aspekte.
% Dies legt einen möglichen Zusammenhang zwischen Vorbildern und SES und somit einen Einfluss auf soziale Ungleichheit und soziale Mobilität nahe, was eine Untersuchung des Zusammenwirkens von Vorbildern und SES nahelegt.
% In dieser Arbeit werden Ergebnisse einer Befragung von Jugendlichen zu ihrem SES und ihren selbstgewählten prominenten Vorbildern mit computerlinguistischen Analysen von Online-Zeitungsberichten über diese Vorbilder kombiniert. Dabei wird verglichen, wie sich die Berichte über von Jugendlichen mit niedrigem und hohem SES benannten Vorbilder im Hinblick auf Themen und Konnotationen unterscheiden. Es wurden signifikante Unterschiede in den Verteilungen der Themen der Zeitungsartikel sowie der darin vermittelten Konnotationen zu Gefühlslagen, Erfolg, prosozialem Verhalten zwischen den niedrigem und hohem SES gefunden.

\glsresetall
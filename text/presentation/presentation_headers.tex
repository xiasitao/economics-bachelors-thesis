\documentclass[
  utf8,
  aspectratio=169,
]{beamer}

% === Theme ===
\usetheme{metropolis}

% === Fonts ===
%\usefonttheme{professionalfonts} % overwrite beamer settings
\DeclareUnicodeCharacter{2009}{\,} 
\usefonttheme[]{serif}
\usepackage{noto}
\usepackage{ebgaramond}
%\usepackage[scale=0.75]{sourcecodepro}
\usepackage[cmintegrals,cmbraces]{newtxmath}
\usepackage{ebgaramond-maths}
% Redefining missing symbols
% https://tex.stackexchange.com/questions/215270/can-someone-explain-this-weird-font-behavior-ebgaramond-maths
\makeatletter
  \DeclareSymbolFont{ntxletters}{OML}{ntxmi}{m}{it}
  \SetSymbolFont{ntxletters}{bold}{OML}{ntxmi}{b}{it}
  \re@DeclareMathSymbol{\leftharpoonup}{\mathrel}{ntxletters}{"28}
  \re@DeclareMathSymbol{\leftharpoondown}{\mathrel}{ntxletters}{"29}
  \re@DeclareMathSymbol{\rightharpoonup}{\mathrel}{ntxletters}{"2A}
  \re@DeclareMathSymbol{\rightharpoondown}{\mathrel}{ntxletters}{"2B}
  \re@DeclareMathSymbol{\triangleleft}{\mathbin}{ntxletters}{"2F}
  \re@DeclareMathSymbol{\triangleright}{\mathbin}{ntxletters}{"2E}
  \re@DeclareMathSymbol{\partial}{\mathord}{ntxletters}{"40}
  \re@DeclareMathSymbol{\flat}{\mathord}{ntxletters}{"5B}
  \re@DeclareMathSymbol{\natural}{\mathord}{ntxletters}{"5C}
  \re@DeclareMathSymbol{\star}{\mathbin}{ntxletters}{"3F}
  \re@DeclareMathSymbol{\smile}{\mathrel}{ntxletters}{"5E}
  \re@DeclareMathSymbol{\frown}{\mathrel}{ntxletters}{"5F}
  \re@DeclareMathSymbol{\sharp}{\mathord}{ntxletters}{"5D}
  \re@DeclareMathAccent{\vec}{\mathord}{ntxletters}{"7E}
\makeatother
\renewcommand{\epsilon}{\varepsilon}
\usepackage[]{todonotes}


%\usepackage[osf, sc]{mathpazo}
%\linespread{1.03}
\usepackage{setspace}

\usepackage[T1]{fontenc}
\usepackage{eurosym}
\newcommand{\red}[1]{\colorbox{red}{\textcolor{white}{\textbf{#1}}}}

% === General ===
\usepackage[utf8]{inputenc}
\usepackage{hyperref}
%\usepackage{sfmath}
\usepackage{amsmath}
\usepackage{mathtools}
\usepackage{physics}
\usepackage{siunitx}
\sisetup{per-mode=fraction,fraction-function=\tfrac}
\DeclareSIUnit\gauss{G}
\usepackage[]{braket}
\usepackage{bbm}
\usepackage{threeparttable}
\usepackage{adjustbox}
\usepackage[english]{babel}
\usepackage{eurosym}
\usepackage{epstopdf}
\usepackage{xcolor}
\usepackage{booktabs}
\usepackage{color, colortbl}
\usepackage{array,multirow}
\usepackage{tabularx}
\usepackage{dcolumn}
\usepackage{setspace}
\usepackage{wrapfig}
\usepackage{caption}
\usepackage{subcaption}
\captionsetup{compatibility=false}
%\captionsetup[figure]{labelformat=empty}% redefines the caption setup of the figures environment in the beamer class.
\usepackage{multido}
\usepackage{multicol}
\usepackage{pgf}

\usepackage{listings}
\usepackage{xcolor}
\usepackage[scale=1]{sourcecodepro}
\definecolor{stringcolor}{RGB}{154,91,145}  % rebecca
\definecolor{keywordcolor}{RGB}{80,189,233} % blue
\definecolor{commentcolor}{RGB}{255,0,0}  % red
\lstset{
  basicstyle=\ttfamily\tiny,
  numbers=left,
  breaklines=true,
  language=Python,
  otherkeywords={True,False},
  commentstyle=\color{commentcolor},
  keywordstyle=\color{keywordcolor}\bfseries,
  stringstyle=\color{stringcolor},
  emph={self, csl, ConeShapedCoilArrangement, FieldMap, Collection, Jones_vector},
  emphstyle={\color{keywordcolor}},
}


% === Color theme ===
\makeatletter
\definecolor{lmugreen}{rgb}{0,.58,.25}
\definecolor{lmu-lightgreen}{rgb}{0,.85,.36}
\definecolor{mpqblue}{rgb}{0, .266, 0.371}

\definecolor{thesisred}{RGB}{255,0,0}
\definecolor{thesisgray}{RGB}{71,72,71}
\definecolor{thesisrebecca}{RGB}{105,62,163}
\definecolor{thesisfuchsia}{RGB}{154,91,145}
\definecolor{thesisblue}{RGB}{80,189,233}

\definecolor{hlgray}{gray}{0.8}
\newcolumntype{A}{>{\columncolor{hlgray}}c}
\setbeamercolor{title}{bg=thesisgray,fg=white}
\setbeamercolor{block title}{bg=thesisgray,fg=white}
\setbeamercolor{frametitle}{bg=thesisgray,fg=white}
\setbeamercolor{normal text}{bg=white,fg=black}
\setbeamercolor{alerted text}{fg=thesisgray}
\setbeamercolor{example text}{fg=green!50!black}
\setbeamercolor{structure}{fg=thesisgray}
\setbeamercolor{background canvas}{parent=normal text}
\setbeamercolor{background}{parent=background canvas}
\setbeamercolor{palette primary}{fg=yellow,bg=yellow}
\setbeamercolor{palette secondary}{use=structure,fg=structure.fg!100!green}
\setbeamercolor{palette tertiary}{use=structure,fg=structure.fg!100!green}
\makeatother

% \AtBeginSection[]{
%   \begin{frame}
%   \vfill
%   \centering
%   \begin{beamercolorbox}[sep=8pt,center,shadow=true]{title}
%     \usebeamerfont{title}\insertsectionhead\par
%   \end{beamercolorbox}
%   \vfill
%   \end{frame}
% }

% === Glossary ===
\usepackage[acronym]{glossaries}
\makeglossaries
\newacronym{bert}{BERT}{Bidirectional Encoder Representations from Transformers}
\newacronym{lda}{LDA}{Latent Dirichlet allocation}
\newacronym{nlp}{NLP}{natural language processing}
\newacronym{pca}{PCA}{principal component analysis}
\newacronym{ses}{SES}{socioeconomic status}
\newacronym{tsne}{tSNE}{$t$-distributed stochastic neighbor embedding}